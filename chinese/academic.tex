% !TeX root = main.tex

\cvsection{学术科研}

\begin{cventries}
  \cventry{导师: 徐枫}{基于物理模拟的 3D 人脸重建}{清华大学软件学院}{2023.04 -- 今}{
    传统 LBS 方法是通过控制表面网格的形状来实现的, 其控制方式是通过骨骼关节位置和权重实现的.
    但是, 在处理复杂的骨骼和面部形状时, 这种方法容易产生与解剖学或动力学不符的结果.
    因此, 该项目旨在使用物理方法模拟骨骼和肌肉的物理行为, 预测人脸表面, 以更好地重建 3D 人脸.
  }
  \cventry{导师: 白家驹}{移动端本地代码库测试}{清华大学计算机系}{2022.06 -- 2023.02}{
    % 出于性能考虑, 移动端的高层应用程序 (比如 Java 编写的安卓程序) 会调用底层本地代码库 (主要由 C/C++ 编写的 so 库).
    % 一旦本地代码库出现缺陷, 将会导致很多移动端应用受到影响, 出现严重的安全问题.
    % 本项目将在研究组已有的工作基础上, 设计和研发创新的测试方法, 来自动生成高层应用程序性代码片段, 并利用异常注入和并法分析等技术, 有效测试现有移动端本地代码库.
    为了提高性能, 移动端的高层应用程序会调用底层本地代码库, 主要由 C/C++ 编写的 so 库.
    当本地代码库出现缺陷时, 会影响许多移动应用并带来安全隐患.
    该项目旨在在已有的工作基础上, 设计和研发创新的测试方法, 以自动生成高层应用程序性代码片段, 并利用异常注入和并发分析等技术, 有效测试现有移动本地代码库.
    \vspace{2.0mm}
  }
  \cventry{导师: 翟季冬}{高性能计算与人工智能}{清华大学计算机系}{2021.03 -- 2022.03}{
    并未做出贡献, 主要专注于学习. \linebreak
    \fullcite{he_fastermoe_2022}
  }
\end{cventries}
